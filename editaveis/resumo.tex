% Resumo em l�ngua vern�cula
\begin{center}
  {\Large{\textbf{\imprimirtitulo}}}
\end{center}

\vspace{1cm}

\begin{flushright}
  Autor: \imprimirautor\\
  \imprimirorientadorRotulo~: \imprimirorientador
\end{flushright}

\vspace{1cm}

\begin{resumo}[RESUMO]
  O resumo deve apresentar de forma concisa os pontos relevantes de um texto, fornecendo uma vis�o r�pida e clara do conte�do e das conclus�es do trabalho. O texto, redigido na forma impessoal do verbo, � constitu�do de uma seq��ncia de frases concisas e objetivas e n�o de uma simples enumera��o de t�picos, n�o ultrapassando 500 palavras, seguido, logo abaixo, das palavras representativas do conte�do do trabalho, isto �, palavras-chave e/ou descritores. Por fim, deve-se evitar, na reda��o do resumo, o uso de par�grafos (em geral resumos s�o escritos em par�grafo �nico), bem como de f�rmulas, equa��es, diagramas e s�mbolos, optando-se, quando necess�rio, pela transcri��o na forma extensa, al�m de n�o incluir cita��es bibliogr�ficas.
  \vspace{\onelineskip}

  \noindent
  \textit{Palavras-chave}: Palavra-chave 1, Palavra-chave 2, Palavra-chave 3.
\end{resumo}